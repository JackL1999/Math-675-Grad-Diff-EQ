\documentclass[10pt]{article}                         
\pagestyle{plain} %-- Document Set Up Command

%-- Document Setup & Layout
\usepackage{fancyhdr}
\usepackage{comment}
\usepackage{multicol}
\usepackage[
  letterpaper,
  left=0.80in,
  right=0.80in,
  textheight=9.5in,
  bmargin=0.75in  % Adjust this value to push the footer down
]{geometry}

%-- Color & Text Formatting
\usepackage[svgnames]{xcolor} 
\usepackage[dvipsnames]{xcolor}

\usepackage{textcomp}
\usepackage{ulem}
\usepackage{graphicx}

%-- Math Packages
\usepackage{amsmath}     
\usepackage{amsfonts}
\usepackage{amstext}
\usepackage{amssymb}
\usepackage{latexsym}
\usepackage{bm}

%-- Graphics and Figures
\usepackage{graphicx}    
\usepackage{tikz}
\usepackage{pgfplots}
\usepackage{circledtext}
\usepackage{float}
\usepackage{caption}
\usepackage{subcaption}

%-- Tables
\usepackage{array}      
\usepackage{tabularx}
\usepackage{tcolorbox}

%-- Lists
\usepackage{enumitem}    
%\usepackage{enumerate}
\usepackage{tasks}
\tcbuselibrary{theorems}
\usepackage{hyperref}
\usetikzlibrary{shapes.geometric} 
\usetikzlibrary{arrows.meta}

%-- Header/Footer
\pagestyle{fancy}
\fancyhf{} % Clear all header and footer fields
\fancyhead[L]{Your Name} % Left header with name
\fancyhead[R]{October 14th 2025} % Right header with date
\renewcommand{\headrulewidth}{0.4pt} % Horizontal line below the header

\newenvironment{AndreaBox}[1]{
\begin{tcolorbox}[
    title=#1,
    colback=white,
    colframe=Fuchsia,
    coltitle=white,
    fonttitle=\bfseries,
    colback=white,
    rounded corners,
    boxrule=0.5pt,
    top=1mm, bottom=1mm, left=2mm, right=-2mm, % right padding creates the long bar
    rounded corners,
    boxrule=0.5pt,
] \raggedright
}{\end{tcolorbox}}

\begin{document}

% Main title
\begin{center}
    \Large \textbf{Comprehensive Exam Notes} \\
    \vspace{0.2cm}
    \normalsize Here we go!
\end{center}

\section*{Section 1: Graduate Algebra}

\section*{Section 2: Real Analysis}
\begin{AndreaBox}{WEEK 1/2: Convergence}
    \begin{itemize}[label={}, leftmargin=1mm, nosep]
        \item {\color{RoyalBlue}\textbf{Learning Goal 1}}
        \item State and check the definitions of pointwise and uniform convergence for sequences and series of functions.
        
        \item {\color{RoyalBlue}\textbf{Learning Goal 2}}
        \item Apply the Weierstrass $M$-test.
        
        \item {\color{RoyalBlue}\textbf{Learning Goal 3}}
        \item Prove and use the theorem relating continuity and uniform convergence.
        
        \item {\color{RoyalBlue}\textbf{Learning Goal 4}}
        \item State and check the definition of uniform continuity.
    \end{itemize}
\end{AndreaBox}

\begin{AndreaBox}{WEEK 4: Fourier Series Part 1}
    \begin{itemize}[label={}, leftmargin=1mm, nosep]
        \item {\color{RoyalBlue}\textbf{Learning Goal 5}}
        \item Describe how we derive the formulas for computing the coefficients of the Fourier series.
        
        \item {\color{RoyalBlue}\textbf{Learning Goal 6}}
        \item Compute the Fourier series for simple examples. “Simple” here means that \\
                the integrals required to compute the coefficients don’t take forever to compute.
    \end{itemize}
\end{AndreaBox}

\begin{AndreaBox}{WEEK 4: MISSING FOURIER SERIES PART 2}

\end{AndreaBox}

\begin{AndreaBox}{WEEK 8: The Space $L^2$: Part 1}
    \begin{itemize}[label={}, leftmargin=1mm, nosep]
        \item {\color{RoyalBlue}\textbf{Learning Goal 7}}
        \item State the definition of $L_2([a,b])$. Check if a given function is or is not in $L_2([a,b])$.
        
        \item {\color{RoyalBlue}\textbf{Learning Goal 8}}
        \item State and prove the Cauchy-Schwartz inequality and the Minkowski inequality.
    \end{itemize}
\end{AndreaBox}

\begin{AndreaBox}{WEEK 9: The Space $L^2$: Part 2}
    \begin{itemize}[label={}, leftmargin=1mm, nosep]
        \item {\color{RoyalBlue}\textbf{Learning Goal 9}}
        \item State, check, and use the definition of a norm.
        
        \item {\color{RoyalBlue}\textbf{Learning Goal 10}}
        \item Identify complete normed vector spaces and explain why a space is or is not a complete normed vector space.
    \end{itemize}
\end{AndreaBox}

\begin{AndreaBox}{WEEK 10: Linear Transformations}
    \begin{itemize}[label={}, leftmargin=1mm, nosep]
        \item {\color{RoyalBlue}\textbf{Learning Goal 11}}
        \item State the definition of a bounded linear operator. Use the definition in proofs.
    \end{itemize}
\end{AndreaBox}
\section*{Section 3: Numerical Analysis}

\section*{Section 4: Complex Analysis}

\section*{Section 5: Topology}

\section*{Section 6: Differential Equations}


\end{document}